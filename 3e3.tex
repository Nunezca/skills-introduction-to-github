\documentclass[]{article}
\usepackage{amsmath,amssymb}
\usepackage{graphicx}
%opening
\title{Identidad de Bézout}
\author{Joaquin Nuñez}

\begin{document}

\maketitle



\section{Identidad de Bezout}
Sean $a, b \in \mathbb{Z}$ no ambos nulos y sea $d=(a:b)$. Entonces, $\exists$ $x, y \in \mathbb{Z}$ tal que $d=ax + by$.


\section{Demostración}

Consideremos el siguiente conjunto $S$

$$
	S =\{xa+yb: x,y \in \mathbb{Z}, ~ xa+by>0\}	
$$

Notemos que como alguno de los números $a,-a,b,-b$ es positivo, tenemos que ese número pertenece a $S$. Pues, por ejemplo, si $a$ es positivo, eligiendo $x=1$ e $y=0$, tenemos que $xa+by > 0$ con $x,y \in \mathbb{Z}$. De este modo, $S$ es un conjunto no vacio. Luego, como $S$ es un subconjunto no vacio de $\mathbb{N}$, sabemos que tiene un elemento mínimo. Sea $d$ el elemento mínimo de $S$. Como $d \in S$, $\exists ~ x,y \in \mathbb{Z}$ tales que $d=xa+yb$

Sean ahora $q$ y $r$ el cociente y resto de la división de $a$ por $d$, es decir, $a=qd+r$ con $0 \leq r < d$. Si $r \neq 0$ tenemos que $r \in S$, pues

$$
r=a-qd=a-q(xa+yb)=(1-qx)a+(-qy)b
$$
en donde $1-qx$ y $-qy$ son números enteros. Pero esto es una contradicción, pues dijimos que $d$ era el mínimo pero $r <d$. Luego $r=0$, con lo que $d \mid a$

Sean ahora $q'$ y $r'$ el cociente y resto de la división de $b$ por $d$, es decir, $b=q'd+r'$ con $0 \leq r' < d$. Si $r' \neq 0$ tenemos que $r \in S$, pues

$$
r=b-q'd=b-q'(xa+yb)=(-q'x)a+(1-q'y)b
$$
en donde $1-q'y$ y $-q'x$ son números enteros. Pero esto es una contradicción, pues dijimos que $d$ era el mínimo pero $r' <d$. Luego $r'=0$, con lo que $d \mid b$

Así, tenemos que $d$ es un divisor común de $a$ y $b$. Falta ver que es el mayor de entre todos los divisores comunes.

Sea ahora $e$ un divisor común cualquiera de $a$ y $b$. Entonces, tenemos que

$$
\begin{cases}
	e \mid a \\
	e \mid b
\end{cases}
\Rightarrow
\begin{cases}
	e \mid xa \\
	e \mid yb
\end{cases}
\Rightarrow
e \mid xa+yb=d
$$
Luego, como $d \neq 0$, tenemos que $e \leq d$. Esto quiere decir que cualquier divisor común de $a$ y $b$ es menor o igual que $d$, lo cual nos dice que $d$ es el mayor de los divisores comunes, con lo que $d=(a:b)$, con lo que $\exists$ $x, y \in \mathbb{Z}$ tal que $d=ax + by$  Así, queda demostrada la identidad de Bezout. $\square$

\section{Referencias}
\begin{enumerate}
	\item Mariano Suárez-Álvarez. Notas de Álgebra. 2021.
\end{enumerate}
\end{document}